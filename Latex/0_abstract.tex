

\renewcommand{\abstractname}{Abstrakt}
\begin{abstract}
Ky punim diplome shtjellon aplikimin e programimit paralel për të gjeneruar fraktalet. Fraktalet, për shkak të natyrës së tyre rekursive, paraqesin sfidë kompjutimi për iterime të larta, duke i bërë ato kandidatë idealë për përpunim paralel. Duke përdorur platformën CUDA me grupet kooperative, llogaritjet përshpejtohen dukshëm në krahasim me metodat tradicionale sekuenciale. Libraria OpenGL është përdorur për të vizualizuar fraktalet, dhe interoperabiliteti CUDA-OpenGL është shfrytëzuar me qëllimin që llogaritjet e shpejtuara në GPU të shfaqen direkt në OpenGL pa nevojën e transferimit të të dhënave në CPU.\\

\noindent \textbf{\textit{Fjalët Kyqe:}} \textit{fraktal, paralel, sekuencial, CUDA, OpenGL, iterim, rekursion, graf, kernel, memorie.}
\end{abstract}



\newpage
\renewcommand{\abstractname}{Abstract}
\begin{abstract}
This thesis elaborates on the application of parallel programming to generate fractals. Fractals, due to their recursive nature, present a computational challenge for higher iterations, making them ideal candidates for parallel processing. Using NVIDIA's CUDA platform with cooperative groups, calculations are significantly accelerated compared to traditional sequential methods. The OpenGL library is used to visualize fractals, and CUDA-OpenGL interoperability is used to allow GPU-accelerated calculations to be rendered directly in OpenGL without the need to transfer data to CPU. \\

\noindent \textbf{\textit{Keywords :}} \textit{fractal, parallel, sequential, CUDA, OpenGL, iteration, recursion, graph, kernel, memory}

\end{abstract}